\documentclass[../main.tex]{subfiles}
\begin{document}
    \chapter{Introduction}\label{ch:introduction}

    \section{PHP}
        PHP\footnotemark is a server-side programming language created by Rasmus Lerdorf in 1995.
            \footnotetext{http://php.net}
        The original name `Personal Home Page' changed to `PHP: Hypertext Preprocessor' in 1998.
        PHP source files are executed using the PHP Interpreter\footnote{https://github.com/php/php-src}. 
        The language is dynamically typed and allows objects to be changed during run-time.
        PHP uses duck-typing which means that there is no object type validationchecking, but only validates if the attempted operation is permitted on the object.
        
        \paragraph{Evolution}
        The programming language PHP evolved since its creation in 1995.
        The first milestone was in the year 2000 when Object-Oriented (OO) language structures were added to the language with the release of PHP 4.0.
        The 5th version of PHP, released in 2004, provided an improved OO structure  including the first type declarations for function parameters.
        Namespaces were added in PHP 5.3 in 2009, to resolve class naming conflicts between libraries and to create better readable class names.
        The OPcache extension is added was added in PHP 5.5 and speeds up the performance of including files on run-time by storing precompiled script byte-code in shared memory.
        The latest 5.x version is 5.6 and includes more internal performance optimisations and introduces a new debugger.
        The most recent stable version is 7, which is not taken into this research.
        In this latest version they achieved a mayor performance increase and memory decrease.
        Also type hints for scalar types are added, method/function return types, and strict typing can be enforced for the available type hints.
       
        \paragraph{Popularity}
        According to the Tiobe Index\footnotemark of July 2016, PHP is the 6th most popular language of all programming languages.
        \footnotetext{http://www.tiobe.com/tiobe\_index, July 2016}
        The language has been in the top 10 since its introduction in the Tiobe Index in 2001.
        More than 80 precent of the websites have a running php backend\footnotemark.
            \footnotetext{http://w3techs.com/technologies/details/pl-php/all/all, July 2016}
        The majority of these websites use PHP version 5, rather than older or newer versions.
        We therefor focus our type inference on PHP version 5.

    \section{Position}
    	% todo check this section
        Due to various dynamic features in PHP not all types and execution paths can be resolved without actually executing the program.
        Source code analysis tools need to know execution paths and types of expressions for optimal results in discovering security vulnerabilities or bugs.
        Such tools could also provide code completions or do automatic transformations when executing refactoring patterns.
        Source code analysis is performed statically or dynamically or a combination of the two.
        In static analysis the program is not executed.

    \section{Contribution}        
        This research contributions to the static analysis research field of dynamic programming languages by presenting a constraint based type inference analysis that over-approximates runtime values at compile time.
        Results of this analysis can be used to improve software analysis tools.
        In order to resolve the types of expressions we implemented a generic model, $M^3$, which holds various program facts for PHP programs.
        
        The main contributions of this thesis are:
        \begin{itemize}
            \item $M^3$ model for PHP programs
            \item constraint based type inference for PHP programs
        \end{itemize}

		An \textit{$M^3$ model for PHP programs} contains various facts about the programs.
		The $M^3$ model, see section \ref{sec:m3_for_php} for more information, was initially solely supporting Java programs.
		The addition of support for PHP programs was beneficial not only for this thesis, but also for other researchers.
		The model provides program context which is used during constrain extraction and solving.
		The model helped other researchers by providing program context when comparing the PHP programming usage with programs written in other languages.

		The biggest contribution of this thesis is the \textit{constrained based type inference for PHP programs}.
		In this type inference process we use an abstract syntax tree to generate type constraints on language constructs.
		We then solve these constraints to come to a set of types for each expression.
		We variated the inference process by adding context information of type annotations and php built-in to the constraint extraction to find out that this helps to resolve more types.
		The inferred types can help IDE tools and programmers by providing helpful tools, which could lead to better software development.        
    
    \section{Plan} 
        The rest of this thesis is organised as follows:
        
        \paragraph{}
        Chapter \ref{ch:background} contains background information and related work.
        The background information consists of important language constructs, information about annotations in PHP, brief introduction to Rascal, $M^3$, and type systems. 
        We end this chapter with related researches and their relation to this research.
        
        \paragraph{}
        Chapter \ref{ch:inference_design_context}, research context, describes the research approach and context.
        We explain under which assumptions this research is executed.
        
        \paragraph{}
        Chapter \ref{ch:inference_design} describes the design of the type inference for PHP.
        We present the constraint rules on various language constructs.
        Next to that we give information about type annotations and php built-in information. 
        
        \paragraph{}
        Chapter \ref{ch:inference_implementation} contains implementation details.
        We show how we implemented the $M^3$ model for PHP.
        Next we explain how we implemented the constraint extracting.
        The constraint solving is explained by showing the used algorithm.
        
        \paragraph{}
        Chapter \ref{ch:evaluation} shows type inference results on real world PHP programs.
        We present the results of multiple programs and analyse the results.
        
        \paragraph{}
        Chapter \ref{ch:conclusion} ends this thesis with the conclusion, future work section and lists various threats to validity.
        
        
\end{document}