\documentclass[../main.tex]{subfiles}
\begin{document}
    \chapter{Introduction}\label{chap:introduction}

    \section{PHP} % Definition/Problem statement: what is the (not)? the topic (Tell something about the current position of PHP. It's become a modern language with features in PHP 5.[3-5].)
        % php has a stable position in the top 10 for 10 years.
        PHP\footnotemark is a server-side scripting language created by Rasmus Lerdorf in 1995.
            \footnotetext{http://php.net}
        The original name `Personal Home Page' changed to `PHP: Hypertext Preprocessor' in 1998.
        PHP source files are executed using the PHP Interpreter. 
        The language is dynamically typed, which means that the types of variables are examined during run-time.
        In statically typed languages all variable types are known at compile time.
        PHP supports duck-typing, which allows variables to change types during execution.
        
        \paragraph{Evolution}
        The programming language PHP evolved after its creation in 1995.
        In the year 2000 Object-Oriented (OO) language structures were added to the langue with the release of PHP 4.0.
        The 5th version of PHP was released in 2004 and provided an improved OO structure.
        Namespace were added in PHP 5.3 in 2009, to be able to resolve class naming conflicts between library and create better readable class names.
        Namespaces are comparable to packages in Java.
        OPcache extension is added was added in PHP 5.5 and speeds up the performance of including files on run-time by storing precompiled script byte-code in shared memory.
        The most recent stable version is 5.6 and includes more internal performance optimisations and introduces a new debugger.
        % The next version of PHP will be 7.
       
        \paragraph{Popularity}
        According to the Tiobe Index\footnotemark of December 2014, PHP is the 6th most popular language of all programming languages.
            \footnotetext{http://www.tiobe.com/index.php/content/paperinfo/tpci/index.html, December 2014}
        The language has been in the top 10 since it's introduction in the Tiobe Index in 2001.
        More than 80 precent of the websites have a php backend\footnotemark.
            \footnotetext{http://w3techs.com/technologies/details/pl-php/all/all, December 2014}
        The majority of these websites use PHP version 5, rather than version 4 or older versions.
        It is therefor wise to focus on PHP version from 5 and discard the older unused versions.
            
        \paragraph{Analysability}
        Although the popularity for more than a decade, there is still a lack of good PHP code analysis tools.
        Tools can help to reveal security vulnerabilities or find vulnerabilities or bugs in source code.
        The tools can also provide code completions or do automatic transformations which can be used to execute refactoring patterns.
        Source code analysis can be performed statically or dynamically or a combination of the two.
        More information on the analysability of php can be found in section \ref{sec:background_language-constructs}.
        
    \section{Position} % Position: what makes this paper unique, position to related work. (Explain about the current position of PHP.)
        In this research we investigate how we can improve the analysability of PHP programs.
        We will show that the use of annotated source code can help to improve the analysability.
        The correctness of the annotations can also be examined by checking the implementation of the code. 
        \\
        %todo rewrite (introduction:position)
        These annotations can help to improve the analysis.
        The results can be used to find security issues, and if they are highly reliable we can even make compiler optimisations.
        \\
        As far as we know, there is no constraint based type inference research like this one performed for PHP.
        That makes this research unique.
        There have been similar analysis for other dynamic languages, like smalltalk, ruby and javascript, but none like this.
       
    \section{Contribution} % Contribution: what is the return of investment for the reader. (Resolving types can: helpt to build IDE tools, identify dependencies, beter security scans, gecombineerd worden met andere analyses (voor future work).)
        % TODO Review (intoduction:contribution) when the result of the analysis are performed.
        TODO Review this part when the result of the analysis are performed.
        \begin{itemize}
            \item Created an M3 for php.
            \item Constraint system.
            \item Show the value of annotations for analysis.
        \end{itemize}

        Some idea's are that this analysis can help IDE tools to perform transformations on the source code.
        (But the performance may not be sufficient.)
        \\
        The creation of the M3 model can help to compare researchers compare PHP programs with other programming languages. For now only Java is implemented, but more can follow (unchecked statement).
    
    \section{Plan} % Plan: outline of the paper. (In section x stuff.) 
        The rest of this thesis is as follows:
        chapter \ref{chap:background} contains background and related work.
        Background exists of important language constructs, information about annotations in PHP, introduction to $M^3$ and type systems. 
        The last section of this chapter shows similar research and their relation to this research.
        Chapter \ref{chap:research_method}, research method, defines the research question and the research context.
        Chapter \ref{chap:research} describes the performed actions of this research. 
        The analysis is presented in chapter \ref{chap:analysis}, with the results in chapter \ref{chap:results}.
        %todo: chapter 7: case study is missing here
        This thesis ends with the conclusions in chapter \ref{chap:conclusion}.
        
        
\end{document}