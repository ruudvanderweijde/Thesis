\documentclass[../main.tex]{subfiles}
\begin{document}
    \chapter{Introduction}\label{chap:introduction}

    \section{PHP} % Definition/Problem statement: what is the (not)? the topic (Tell something about the current position of PHP. It's become a modern language with features in PHP 5.[3-5].)
        % php has a stable position in the top 10 for 10 years.
        PHP\footnotemark is a server-side scripting language created by Rasmus Lerdorf in 1995.
            \footnotetext{http://php.net}
        The original name Personal Home Page changed to PHP: Hypertext Preprocessor in 1998.
        PHP source files are executed using the PHP Interpreter. 
        The language is dynamically typed, which means that the behaviour of the source code will be examined during run-time.
        Statically typed languages would apply these modification during compile type.
        PHP supports duck-typing, which means that the type of an expression can be transformed to another type at a certain point.
        
        \paragraph{}%Evolution}
        The programming language PHP evolved after its creation in 1995.
        In the year 2000 Object-Oriented (OO) language structures were added to the langue with the release of PHP 4.0.
        The 5th version of PHP was release in 2004 including improved the OO support.
        To be able to resolve conflicts between library and create better readable class names, namespaces were added to the release of PHP 5.3 in 2009.
        Namespaces are comparable to packages in JAVA.
        The most recent stable version is 5.5 in which the OPcache extension is added. 
        OPcache speeds up the performance of including files on run-time by storing precompiled script bytecode in shared memory.
       
        \paragraph{}%Popularity}
        According to the Tiobe Index\footnotemark of july 2014, PHP is the 7th most popular programming language.
            \footnotetext{http://www.tiobe.com/index.php/content/paperinfo/tpci/index.html, July 2014}
        The language has been in the top 10 since its introduction in the Tiobe index in 2001.
        More than 80 procent of the websites have a php backend\footnotemark.
            \footnotetext{http://w3techs.com/technologies/details/pl-php/all/all, July 2014}
        The majority of these websites use PHP version 5, rather than version 4 or version 3.
        It is therefor useful to focus on PHP version from 5 and disgard the older unsupported versions.
            
        \paragraph{}%Analysability}
        Although the popularity for more than a decade, there is still a lack of good PHP code analysis tools.
        These tools can help to reveal security vulnerabilities or bugs in source code.
        The tools can also provide code completions to developers or make automatic transformations on the code possible, for example to execute refactoring patterns.
        Other dynamic languages suffer the same difficulties
        
    \section{Position} % Position: what makes this paper unique, position to related work. (Explain about the current position of PHP.)
        %TODO fill position, reread after writing
        As far as we know, there is no constraint based type inference research like this one performed for PHP.
        That makes this research unique.
        There have been similar analysis for other dynamic languages, like smalltalk, ruby and javascript.
       
    \section{Contribution} % Contribution: what is the return of investment for the reader. (Resolving types can: helpt to build IDE tools, identify dependencies, beter security scans, gecombineerd worden met andere analyses (voor future work).)
        % TODO rewrite this piece
        TODO Review this part when the result of the analysis are performed.
        Some idea's are that this analysis can help IDE tools to perform transformations on the source code.
        But the performance may not be sufficient.
        \\
        \blindtext % TODO remove blindtext
    
    \section{Plan} % Plan: outline of the paper. (In section x stuff.) 
        The rest of the thesis is as follows:
        chapter \ref{chap:background} contains background and context information about related work.
        Here we will explain the php language and explain similar research.
        In the next chapter \ref{chap:research_method} the research method is explained, which will explain the steps taken in this the research.
        
\end{document}