\documentclass[../main.tex]{subfiles}
\begin{document}
    \chapter{Research Method}\label{chap:research_method}

    \section{Introduction}
    $<<$ Some kind of introduction here... $>>$

    \section{Research question}
    The research question will be something like: \\
    \begin{quote}
        Will the use of annotations\footnotemark{} provide more resolution for constraint based type inference?
        \footnotetext{The annotations are limited to: \texttt{@param}, \texttt{@return}, \texttt{@var}, and \texttt{@inheritDoc}.}
    \end{quote}
    Subquestions:
    \begin{quote}
        - Where do the differences come from?
        \\
        - Which differences have a positive influence on the results?
        \\
        - Which differences have a negative influence on the results?
        \\
        - Can we say something about the reliability of the annotations?
    \end{quote}
    * These questions need to be more concrete, making them `testable'.
        
    \section{Research context}
    The following items should be wrapped in a piece of text. (Design decisions??)
    \begin{itemize}
        \item We assume that the program is correct. This means that warnings can happen, but fatal errors not.
        \item We assume that all files are included.
        \item We assume that register globals is off! (maybe add some other runtime environments)
        \item Ingore warnings (because most production code has them off)
        \item Our analysis is flow-, control-, and context-insensitive (explain what this means)
    \end{itemize}
    These items will not be covered by the analysis (maybe add this to threats/future work)
    \begin{itemize}
        \item Analysis is flow insensitive
        \item Closure
        \item References
        \item Variable constructs (variable -variable, -method/function calls, -class instantiation, eval) :: todo: explain WHY not.
        \item Yields
    \end{itemize}

\end{document}
