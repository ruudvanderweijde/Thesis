\documentclass[../main.tex]{subfiles}
\begin{document}
    \section{Research Method}\label{sec:research_method}
    This section will contain:
    \begin{itemize}
        \item TODO: Explain the used method. %TODO research method
        \item TODO: Hypothese/research questions %TODO research question
    \end{itemize}
    
    Steps taken from project plan:
    \begin{itemize}
        \item Implement in PHP-Parser/Rascal in the following steps:
        \item Find out how the compiler works; check control flow graphs, check limitations (part of literature study).
        \item Convert AST to M3 model for PHP.
        \item Convert M3 to OFL.
        \item First OFL step is context-insensitive.
        \item - Optional: make OFL context-sensitive.
        \item Apply flow propagation algorithm, as described in the book “Reverse Engineering of Object Oriented Code”
        \item Define and add taint restrictions (what can be labeled as tainted?)
        \item Define and add untain restrictions (how will tainted values be untainted)
        \item Add application based settings to optimize results
        \item `Include` optimization (try to resolve more includes using the study of mark hills)
        \item Object construction optimization using annotations.
        \item YAML optimizations to improve analysis results.
        \item YAML can be used to scope the number of input flows.
        \item Validation of the analysis results:
        \item Use pre-defined files analyzed in different tools.
        \item Use analyzed tools in similar research that is still available for download.
        \item Compare results similar existing tools: Pixy, RIPS, SAFERPHP. (All these tools focus on SQLi and XSS).
        \item - Optional: Try to run the code in the sinks using a PHP interpreter.
        \item Produce final results (this step will be iterated)
    \end{itemize}

\end{document}
