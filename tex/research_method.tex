\documentclass[../main.tex]{subfiles}
\begin{document}
    \chapter{Research Method}\label{chap:research_method}

    In this research we will try to answer research questions.
    The questions will be described in section \ref{sec:research_question}
    The research context is described in section \ref{sec:research_context}

    \section{Research question}\label{sec:research_question}
    The research question will be something like: \\
    \begin{quote}
        Will the use of annotations\footnotemark{} provide more resolution for constraint based type inference?
        \footnotetext{The annotations are limited to: \texttt{@param}, \texttt{@return}, \texttt{@var}, and \texttt{@inheritDoc}.}
    \end{quote}
    Subquestions:
    \begin{quote}
        - Where do the differences come from?
        \\
        - Which differences have a positive influence on the results?
        \\
        - Which differences have a negative influence on the results?
        \\
        - Can we say something about the reliability of the annotations? (What is the accuracy of the results: recall and precision)
    \end{quote}
    * These questions need to be more concrete, making them `testable'.
    \\
    Explain the subquestions here.
    
    \section{Research context} \cite{sec:research_context}
    In order to let our research take place, we need to make sure that some environment variables are constant.
    
    \paragraph{Program Correctness}
    In order to be able to execute this research we will need to assume that the program is correct and works as intended. We will assume that the system contains no bugs.
    
    \paragraph{File includes}
    In this research we will assume that all file are included during runtime. 
    When a PHP system is constructed of classes with namespaces, the files will be logically loaded using PHP's autoloader.
    Because most recent systems use namespaces, we will assume that all files are included.
    For legacy systems, this can influence the results of this research.
    
    \paragraph{Register globals}
    Register globals allows variables to be magically be created from GET and POST values.
    Since it is discouraged to use this setting, we will assume that all software products have this setting disabled.
    
    \paragraph{PHP Warnings}
    For this research we will ignore all warnings
    
    The following items should be wrapped in a piece of text. (Design decisions??)
    \begin{itemize}
        \item We assume that the program is correct. This means that warnings can happen, but fatal errors not.
        \item We assume that all files are included.
        \item We assume that register globals is off! (maybe add some other runtime environments)
        \item Ingore warnings (because most production code has them off)
        \item Our analysis is flow-, control-, and context-insensitive (explain what this means)
    \end{itemize}
    These items will not be covered by the analysis (maybe add this to threats/future work)
    \begin{itemize}
        \item Analysis is flow insensitive
        \item Closure
        \item References
        \item Variable constructs (variable -variable, -method/function calls, -class instantiation, eval) :: todo: explain WHY not.
        \item Yields
    \end{itemize}

\end{document}
