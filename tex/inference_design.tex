\documentclass[../main.tex]{subfiles}
\begin{document}
    \chapter{Design of PHP type inference}\label{ch:inference_design}

    % todo check introduction
    Todo check/add introduction
    %The research is executed in 3 main steps.
    %The first step creates an $M^3$ model for a PHP program which contains various facts about the program. 
    %The creation of an $M^3$ for PHP is explained in more details in section \ref{sec:m3_for_php}.
    %Once the $M^3$ model is constructed, the second step is to extract constraints from the program using the $M^3$ model. How and which constraints are extracted is described in section \ref{sec:constraint_extraction}.
    %In the third step, in section \ref{:constraint_solving}, the constraints are solved and resolve the types of variables used in the program.
    %The final section \ref{sec:annotations} of this chapter explains how annotations can be included in the process to gain more precise results.
    
    \section{Constraint extraction}\label{sec:constraint_extraction}
    
    %In the previous section \ref{sec:m3_for_php} we've shown how an $M^3$ model for PHP is created.
    In this section we present the constraint definitions and how they are extracted.
    
    \subsection{Constraint definitions}
    The constraint definitions we use are based on the definition of Palsberg and Schwartzbach\cite{Pal:94}.
    We have extended the definition to conform to the PHP language.
    A legend with all symbols is displayed in table \ref{table:constraintLegend}, followed by the constraint definitions for PHP.

% https://en.wikibooks.org/wiki/LaTeX/Linguistics

    \begin{table}[H]
    	\center
        \begin{tabular}{ r c l | r c l } 
        	\toprule
        	symbol & & description & 
        	symbol & & description \\ 
        	\midrule
 
            $\equiv$ & = & equivalent expression &
            $=$     & = & equivalent type \\
            
            $:=$    & = & assignment &
            $C$     & = & a class \\
            
            $<:$    & = & (lhs) is subTypeOf (rhs) &
            $\rightarrow c$     & = & a class constant \\
            
            $E_k$   & = & an expression &
            $\rightarrow p$     & = & a class property \\
            
            $\llbracket{}E_k\rrbracket{}$ & = & typeset of an expression &
            $\rightarrow m$     & = & a class method \\
            
            $f$     & = & a function &
            $\llbracket{}m\rrbracket{}$   & = & (return) type of a method call \\
            
            $\llbracket{}f\rrbracket{}$   & = & (return) type of a function &
            $(A_n)$ & = & the n'th actual argument \\
            
            $::c$   & = & static property fetch &
            $(P_n)$ & = & the n'th formal parameter \\
            
            $::m$   & = & static method call &
            $th$    & = & type hint \\
            
            $::p$   & = & static property fetch &
            $v$     & = & default value \\
            
            Mfs     & = & modifiers &
            $\in$ & = & is defined in \\
                        
            $\{ \}$     & = & set of types &
            \texttt{instanceof} & = & instance of class in parse tree \\
            
            $\land$ & = & conjunction &
            $\lor$   & = & disjunction
            
        \end{tabular}
        \caption{Constraint definition legend}
        \label{table:constraintLegend}
   \end{table}
   
    We write the definitions in the following form:
	   
    \begin{prooftree}
    	\AxiomC{premiss 1} 
    	\AxiomC{premiss 2} 
    	\BinaryInfC{constraint 1,} \noLine
    	\UnaryInfC{constraint 2}
	\end{prooftree}
	
	Above the horizontal line we write the premisses.
	In our case premisses are PHP expressions which are true or false depending on the context.
	If the premiss is true for a PHP statement or expression we can define the constraints below the horizontal line.

    \subsection{Scalars}
    Extracting constraints from the scalar types is straight forward.
    The PHP parser defines the string, integer, and float types in the AST.
    Booleans and null values can also be easily found in the AST.
    \fact{scalar_string}{Strings}{
    	Strings in PHP can be written with single or double quotes.
    	In the AST these are represented by the string type.
    }
    \fact{scalar_integer}{Integers}{
    	In the example below you can see integers with different bases.
    	All of these are parsed into an integer type in the Rascal AST.
    }
    \fact{scalar_float}{Floats}{
    	Floats can have different forms in PHP code, but are all parsed to float objects in the Rascal AST. 
    }
    \fact{scalar_boolean}{Boolean values}{
    	Boolean values in PHP are case sensitive, as you can see in the examples.
    	True and false are reserved keywords in PHP.
    	In the premiss we've only provided the lower case values.
    }
    \fact{scalar_null}{Null values}{
    	Null is a reserved keyword in PHP.
    	When we encounter null in the source code we can add the nullType type constraint.
    }

    %%
    %% Assignments
    %%
    \subsection{Assignments}
	Assign statements transfer values from one expression or variable into another.
	PHP uses the $=$ symbol as assignment syntax.
	In the premiss we use $:=$ for assigns.
	
    \fact{assignment1}{Assignment}{
    	When an assignment is used, we can extract the following constraint: the right hand side ($E_2$) of the assignment is a subtype of the left hand side ($E_1$).
    	This relation is a subtype relation, not an is equal relation, because of the subclass relations of inheritance.
    	The whole expression ($E$) is equal to the newly assigned value.
    }
    \fact{ternary}{Ternary operator}{
    	The \textit{ternary} operator is a conditional assignment.
    	If the expression $E_1$ is evaluated as \texttt{true}, the left hand side ($E_2$) is the value of the whole ternary expression (i).
    	If $E_1$ is evaluated as \texttt{false}, the right hand side ($E_3$) is the value.
    	The constraint we can extract from the ternary expression is that the type of the whole expression should be the type of $E_2$ or $E_3$.
    	\\
    	The ternary operator without a left hand side value (ii), also known as the elvis operator, returns the value of $E_1$ when $E_1$ is evaluated as \texttt{true}.
    	Here the type of the expression should be either the type of $E_1$ or $E_3$.
    }
    \fact{assignment2}{Assignments resulting in integers}{
		PHP provides several assignment statements combined with operators.
		The type of the left hand side ($E_1$) is in the cases of \textit{bitwise and} (i), \textit{bitwise inclusive or} (ii), \textit{bitwise exclusive or} (iii), \textit{bitwise shift left} (iv), \textit{bitwise shift right} (v), and \textit{modulus} (vi) always of \textit{integer} type.
    }
    \fact{assignment3}{Assignment with string concat}{
    	When the \textit{string concat} operator is used, in combination with the assignment operator (i), the type of the left hand side ($E_1$) is always a string.
    	\\
    	About the right hand side ($E_2$) we can say that \textbf{if} the type of $E_2$ is a subtype of object, then this object should have the method \texttt{$\_\_toString()$} (ii).
    }
    \fact{assignment4}{Assignments with division or subtraction operator}{
    	\textit{Division} (i) and \textit{subtraction} (ii) assignment in PHP will always result in an integer type.
    	This is the case for all values, except for array's.
    	A fatal error will occur when the right hand side value is of type array.
    }
    \fact{assignment5}{Assignments resulting in numbers}{
    	The result of an \textit{multiplication} (i) and \textit{addition} (ii) assignment is either a float or an integer.
    	When the type of the right hand side ($E_2$) is either booleanType, integerType, or nullType, the result of the assignment ($E_1$) will be of integerType.
    	If $E_2$ is of any other type, $E_1$ will be of type floatType.
    	Float and integer are both subtypes of integers, so we can use the subtype relation for numberType for this.
    }
    
    %%
    %% Unary operators
    %%    
    \subsection{Unary operators}
	Unary operators in PHP consist of positive and negative numbers, negation operators, and increase and decrease operators.

    \fact{unary_numbers}{Positive and negative numbers}{
    	When a \textit{plus} (i) or \textit{minus} (ii) sign is used in PHP in front of a variable, the type of the whole expression must be of numberType.
    	The type of the variable cannot be of any arrayType.
    }
    \fact{unary_negation}{Negation operators}{
    	The PHP language holds two types of negation operators.
    	The type of the whole expression for \textit{normal negation} operator (i) is boolean.
    	For the \textit{bitwise negation} operator (ii) the type of attached variable is either a number or a string.
    	The type of the whole expression is an integer or string.
    }
    \fact{unary_post_increment}{Post increment operators}{
    	From post increment and decrement operators we can only extract conditional constraints.
    	\\
    	If the type of $E_1$ is of any \texttt{array} type, the result of the expression is also of any \texttt{array} type (i).
    	\\
    	If the type of $E_1$ is of \texttt{boolean} type, the result of the expression is also of \texttt{boolean} type (ii).
    	\\
    	If the type of $E_1$ is of \texttt{float} type, the result of the expression is also of \texttt{float} type (iii).
    	\\
    	If the type of $E_1$ is of \texttt{integer} type, the result of the expression is also of \texttt{integer} type (iv).
    	\\
    	If the type of $E_1$ is of \texttt{null} type, the result of the expression is either of \texttt{integer} or \texttt{boolean} type (v).
    	\\
    	If the type of $E_1$ is of any \texttt{object} type, the result of the expression is also of any \texttt{object} type (vi).
    	\\
    	If the type of $E_1$ is of \texttt{resource} type, the result of the expression is also of any \texttt{resource} type (vii).
    	\\
    	If the type of $E_1$ is of \texttt{string} type, the result of the expression is either of \texttt{number} or \texttt{string} type (vii).
    	\\
    	The rules below are only written for the post increment, but also apply on the post decrement.
    }
    \fact{unary_pre_increment}{Pre increment operators}{
    	From pre increment and decrement operators we can also only extract conditional constraints.
    	The rules are similar to the rules for the post increment, except for the \texttt{nullType()}.
    	If the type of $E_1$ is of \texttt{null} type, the result of the expression is either of \texttt{null} type (v).
    }    
    
    %%
    %% Binary operators
    %%
    \subsection{Binary operators}
	Addition-, subtraction-, multiplication-, division-, modulus-, bitwise-, comparison-, and logical operators are in PHP binary operators.
	
	\fact{binary_addition}{Addition operator}{
		The result of an addition operator will always be a number or an array (i).
		If the left and right hand side are both arrays, the return type will be array (ii).
		In this case two arrays are merged.
		In all other cases the result of this operation is a number (iii).
	}
	\fact{binary_sub_mul_div}{Subtraction multiplication division operators}{
		The \textit{subtraction} (i), \textit{multiplication} (ii), and \textit{division} (iii) operators are merged together in this paragraph because they have identical behaviour.
		The result of these operations is always of \texttt{number} type.
		The operations cannot be used if one of the sides is of type \texttt{array}.
		Therefore we can says that the left and right hand side cannot be of \texttt{array} type.
	}
	\fact{binary_mod_bitshift}{Modulus and bitwise shift operators}{
		The merge of \textit{modulus} (i) and \textit{bitwise shift} (ii, iii) operators seems not so obvious at first, but they have the same behaviour.
		The results of these operations is of integer type.
	}
	\fact{binary_bitwise}{Bitwise operators}{
		The results of the bitwise operators \textit{and} (i, ii, iii), \textit{or}, and \textit{xor} is always of \texttt{integer} or \texttt{string} type.
		When the left and right hand side are both strings, the result of the operation is also of type \texttt{string}.
		In all other cases the result of this operation is a number.
		%For the sake of readability we've omitted the constraints for the operators \textit{bitwise or} (inclusive or, $\vert{}$) and \textit{bitwise xor} (exclusive or, $\^{}$), because they have provide the same results as the \textit{bitwise and} operator.
	}
		
    \fact{comparisonOperators}{Comparison operators}{
    	The result of the comparison operators is always of boolean type.
    	The comparison operators are \textit{equals} (i), \textit{identical} (ii), \textit{not equal} (iii), \textit{not equal} (iv), \textit{not identical} (v), \textit{less than} (vi), \textit{greater than} (vii), \textit{less than or equal to} (viii), and \textit{greater than or equal to} (ix) operators.
    }
    \fact{logicalOperators}{Logical operators}{
    	Just like the comparison operators, the result of the logical operators is always of boolean type.
    	The logical operators are \textit{and} (i), \textit{or} (ii), \textit{xor} (iii), \textit{and} (iv), and \textit{or} (v).
    }
    
    %%
    %% Array
    %%
    \subsection{Array}
    From the PHP parser we get array declaration and array access nodes.
    
    \fact{array}{Array declaration}{
    	From the array declarations (\texttt{array()} or \texttt{[]}) we can extract the constraint that they should be of any \texttt{array} type.
    }
    \fact{arrayAccess}{Array access}{
    	From the usage of array access syntax you cannot tell what the type of the expression is.
    	The same syntax is used to access strings.
    	We can extract that the type of the base expression should not be of object type (i).
    	If we know that the base type is of \texttt{string} type, we know that the result of the expression will also be a string (ii).
    	When the base type is an array, the result type is the type of the elements in there array (iii).
    	For all other cases, when the base type is not an string or array, the result of the expression will be of \texttt{null} type (iv).
    }

    %%
    %% Casts
    %%
    % todo __tostring is now used on a typeset... need to fix it or remove it
    \subsection{Casts}
    \fact{casts}{Casts}{
    	PHP contains syntax to arrays, booleans, integers, floats, objects, strings, and to unset variables.
    	The result of a cast to array is of any \texttt{array} type (i).
    	For casting to boolean there are two keywords, \textit{bool} (ii) and \textit{boolean} (iii), and the result will always be of \texttt{boolean} type.
    	There are three keywords to cast to floats, \textit{float} (iv), \textit{double} (v), and \textit{real} (vi).
    	Casts to integer integer type, you can use \textit{integer} (vii) or \textit{int} (viii) keywords.
    	Any cast to \textit{object} (ix) will result in any \texttt{object} type.
    	A cast to \textit{string} will always result in a \texttt{string} type.
    	String casts (x) will always result in a \texttt{string} type. If we know that the expression ($E_1$) is an object, we know that this method needs to have an \texttt{$\_\_toString()$} method (xi).
    	The last cast, \textit{unset}, results in a \texttt{null} type.
    }
        
    %%    
    %% Clone
    %%
    \subsection{Clone}
    \fact{clone}{Clone}{
    	From the PHP function \textit{clone} we can extract the constraint that the type of the given expression and the result must be of any \texttt{object} type.
    	We also know that the type will not change, and so the type of the expression will be the same as 
    }
    
    %%
    %% Class
    %%
    \subsection{Class}
    This section contains fact extraction rules from object syntax.
    Class instantiation, special keywords, method calls, parameters, and class constants.
    
    \fact{classInstantiation}{Class instantiation}{
    	Classes can be instantiated with the name of the class.
    	The type of the whole expression is then of the specific \texttt{class} type (i).
    	When a class is dynamically instantiated, we only know that it should be of some \texttt{object} type, and that the type of the expression should be any \texttt{object} or \texttt{string} type (ii).	
    }
    \fact{classKeywords}{Special keywords}{
    	PHP contains a few class related reserved keywords with special behaviour.
    	These keywords can be used inside a class scope ($\in C$).
    	From the usage of the keyword \textit{self} we know that the type of the expression should be the same \texttt{class} type as which the keyword is defined in (i).
    	The constraint we can extract from \textit{self} is that the type should be any \texttt{object} type and it should be either the contained class or one of the parent classes.
    	The behaviour of \textit{\$this} (ii) and \textit{static} (iii) differs, but the constraints we can extract are equal to the \textit{self} keyword.
    	The \textit{parent} (iv) keyword differs because it must be a super type of the class they keyword is defined in.
    }
    %\fact{property1}{Class property fetch}
    %    {\\ \footnotesize{* Possible add fact that the field E is declared in class C, when it is on the left side of an assignment.}}
    %\fact{property2}{Class property fetch variable}{}
    \fact{classMethodCall}{Method calls}{
    	From the usage of a method call (\textit{expression} -> \textit{expression}) we can extract the constraint that the type of the left hand side should be an object (i and ii).
    	If the right hand side ($E_2$) is a name of a method, we can extract the constraint that the left hand side ($E_1$) must implement this method (ii).
    	% Could do something with the reserved keywords here (different behaviour with parent::method
    }
    
    %\fact{classConstants}{Class constants}{
    %	Class constants
    %}
    
    %\subsection{Parameters}
    %\fact{parameter1}{Parameters in class instantiation}{
    %    
    %    \footnotesize{*These parameters are just examples for what happens if they have typeHints ($th$), default values($v$) or none}
    %    \footnotesize{*The constructor can be found in the m3 model (@constructors(loc classDecl, loc constructorMethodDecl))}
    %}
    
    %%
    %% Scope
    %%
    \subsection{Scope}
    In order to define the type of a variable within a scope, we have conducted the following constraints.
    \fact{function1}{Variables}{
    	The type of a certain variable is defined by adding an equal constraint on the logical name and the location.
    	The scope of these variables is contained in the logical name.
    	The logical name contains the name of the scope, which is optionally a namespace and the name of a class, method or function, and the name of the variable.
    	An example of a logical name for the variable \texttt{\$a} in the function \texttt{f} in the global namespace is \texttt{|php+functionVar:///f/a}.
    }
    \fact{return}{Return types}{
		The type of a function or method is defined by the return statements it contains.
		When there are no return statements declared in a function or method (i), we can extract the constraint that the function will always return \texttt{nullType}.
		The type of a function is also \texttt{nullType} when there is a return statement without an expression (ii), like \texttt{return;}.
		When there are multiple return statements, the return type of the function or method is the concatenation of the types of the expressions (iii). 
	}
	
    %%
    %% Function calls
    %%
    %\subsection{Function calls}
    %\fact{functionCall1}{Function call}{}
    %\fact{functionCall2}{Function call variable}{}
\end{document}
