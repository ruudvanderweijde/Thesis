\documentclass[main.tex]{subfiles}
\begin{document}
    \chapter{Conclusion}\label{ch:conclusion}
    %todo write this whole chapter
    Summary of the whole work, with conclusions. T.B.A.
    
    \section{Conclusion}
    % we've seen that adding annotations and information about php built-ins helps to resolve types 
    
    \section{Future work}
    These items will not be covered by the analysis (maybe add this to threats/future work)
    \begin{itemize}
        \item Analysis is flow insensitive
        \item Closure
        \item References
        \item Variable constructs (variable -variable, -method/function calls, -class instantiation, eval) :: todo: explain WHY not.
        \item Yields
        \item Traits
    \end{itemize}

    
    \paragraph{}
    Explain something about combining this analysis to other analysis (like dead code elimination, constant folding/propagation resolve, alias analysis, array analysis) to gain more precise results.
    
    \paragraph{}
    Something about performance optimisations... Explain what is already done to boost the performance and what still can be done.
        
    \paragraph{}
    Use a bigger corpus to gains better results of the analysis by doing analysis on more programs.
    
    \section{Threats to validity}	
    % todo check this
    Because we perform an over approximation of run time values at compile time, the results of this research are not 100\% accurate.
	Although we strive to be as accurate as possible, we cannot guarantee the correctness of the results.
		
	\paragraph{Missing constraints}
	By adding a constraint we limit the number of possible types for that expression.
	When we forget to add a constraint, or not able to provide a constraint due to magic features usage, there is a risk that we.
	
	\paragraph{Law of small numbers}
	Because our sample size is small, we cannot generalise these results.
	In order to gain general conclusions we should increase our sample size.
	We can achieve this by running analysis on more software projects.
	
	\paragraph{Object references}
	One thing we do not take into account are the side effects that can be caused by the passing arguments as reference.
	When an referenced argument is modified, the referenced variable is also modified.
	This could theoretically lead to type changes.
	
	\paragraph{PHPDoc correctness}
	In this research we also assume that the provided type annotations in the PHPDoc's are correct.
	These annotations are ignored during program execution and therefore could be incorrect without noticing. 

	
\end{document}
