\documentclass[../main.tex]{subfiles}
\begin{document}
    \chapter{Background and context}\label{chap:background}
    
    \section{PHP Language Constructs}
    In this section important (for this research important!) language constructs are presented.
    Explanations of these constructs should help to understand the performed analysis.
    Including some concepts like scope, includes, dynamic variables, dynamic class instantiation, dynamic function call, dynamic dispatch, runtime environment variables and constants, late static binding (static keyword), magic methods.
    
    \subsection{Scoping}
    PHP has a few scopes. "Define what scope is".
    Global, namespace, class, method, function.
    The global scope is contained in every file which is not inside a function or class.
    The global scope can contain namespaces.
    Namespaces are comparable to packages in Java.
    When namespaces are used, classes and functions will be scoped to the namespace.
    You can access them by providing the namespace name.
    \\
    Todo: say something about the global statement and \$GLOBALS. (it is resolved in the M3 relation @uses). Refer to this link\footnotemark.
        \footnotetext{http://php.net/manual/en/language.variables.scope.php, July 2014}
        
    \begin{itemize}
        \item In general:
        \item Functions are declared in 

    \end{itemize}

    
    \subsection{Includes}
    Note to myself: how will I deal with this concept in my analysis (totally ignore it??? when maybe not add it to this background information. In our research we will assume that all files are included.).
    The problem of including files can be reduced using namespaces and autoloading.
    When a class which is not loaded in memory is instantiated, the autoloading will try to include a file and load the class. 
    For this analysis we will include all files 
    \\
    * Refer to the analysis of mark hills, that most files can be resolved, but not all. 
    We consider the use of including scripts for logic as bad practice. 
    Every file should contain a class, and in this case, it is for our analysis not very interesting to resolve the includes.
    
    \subsection{Conditional functions and classes}
    Explain the code below.
    \lstinputlisting[language=PHP,label=conditionals1,caption=Conditional class and function definitions]{src/php/conditionals1.php}

    Explain the code below.
    \lstinputlisting[language=PHP,label=conditionals2,caption=Conditional function declaration]{src/php/conditionals2.php}

    
    \subsection{Dynamic features}
    These include dynamic variables, dynamic class instantiations, dynamic function calls.
    

    \subsection{Late static binding}
    Late static binding\footnotemark is implemented in PHP since version 5.3 by adding the keyword `static' to the language.
    \footnotetext{http://php.net/manual/en/language.oop5.late-static-bindings.php, July 2014}
    It has the same function as `parent' and `self', because they both point to a class. 
    The main difference is that `parent' and `self' can be resolved statically.
    `static' can only be resolved on runtime and represents the exact class that is instantiated.
    
    \subsection{Magic methods}
    In PHP it is allowed to call methods or use properties that do not exists.
    Normally this would result in a fatal error, but not with the use of magic methods.
    One of the magic methods is het constructor method `\_\_ construct'.

    \subsection{Dynamic class properties}
    Although it is a good practice to define your class properties, it is not required.
    On runtime it is possible to add properties to classes, even without the implementation of magic methods.
    
    \lstinputlisting[language=PHP,label=dynamicProperty,caption=Dynamic class property,label=x]{src/php/dynamicProperty.php}

    \subsection{Annotations}
    Explain here how annotations work in php. (in the next section I will explain something about parsing/reading annotations for the analysis)
    
    \subsection{Other concepts}
    Todo: move this list to a different chapter, this is about my research. Not about php in general.
    \begin{itemize}
        \item We assume that the program is correct. This means that warnings can happen, but fatal errors not.
        \item We assume that all files are included.
        \item We assume that register globals is off! (maybe add some other runtime environments)
        \item Ingore warnings (because most production code has them off)
    \end{itemize}
    These items will not be covered by the anlysis (maybe add this to threats/future work)
    \begin{itemize}
        \item Analysis is flow insensitive
        \item Closure
        \item References
        \item Variable constructs (variable -variable, -method/function calls, -class instantiation, eval) :: todo: explain WHY not.
        \item Yields
        
    \end{itemize}


    
    \section{Rascal}
    Explain what \Gls{Rascal} is and what we use it for...
    Some general stuff here.

    \section{M3}
    In this chapter, only describe what m3 is.
    In some next section I will explain the additional php m3 stuff.
    \\
    But for now (need to create an image of the steps here):
    \begin{itemize}
        \item Parse all files (create \gls{AST}s)
        \item For each file, create an m3 of the ast of the parsed file (aka extract information)
        \item Combine all the m3s
        \item Now run more analysis when all facts are collected in the m3.
        \item Done. M3 is finished! 
    \end{itemize}

    \footnotesize{*Maybe add this section to the next chapter, for the research methods. Many aspects will be useful for the type inference analysis}
    \\
    The M3-model is a generic model which can be used to analyse software programs.
    Our goals is to provide the results in an M3 model.
    Future research can use this to compare different programming languages.
    \\ 
    The following core items are filled for M3:
    \begin{itemize}
        \item Containment
        \item Declarations (need to explain in more details)
        \item Modifiers
        \item Extends
        \item Uses (explain in details what is and what is NOT covered)
    \end{itemize}
    
    The following php specific items are added:
    \begin{itemize}
        \item Extends (class or interface and their extended class or interface)
        \item Implements (which class implements which interfaces)
        \item TraitUses (which class uses which trait)
        \item Parameters (methods and functions and their parameters)
        \item Constructors (which class uses which constructor, explain this more)
        \item Aliases (class aliases, for example the usage of class\_{}alias)
        \item Annotations (contraints annotations on classes, methods, fields and variables)
    \end{itemize}
    Todo, explain in more details...

    
    \section{Related work}
    Describe these:
    \begin{itemize}
        \item `The HipHop Compiler for PHP'\cite{Zhao:12} (not much information available, only source code)
        \item `Phantm: PHP analyzer for type mismatch'\cite{Kne:10,Bar:10} (investigate this in more details)
        \item PHPLint \footnotemark (uses a different kind of annotations, not the java like phpdocs)
        \item `Soft typing and analyses on PHP programs'\cite{}, code implementations: https://github.com/henkerik/typing and https://github.com/marcelosousa/soft-typing-PHP5 (created for php4, code for php5, should check this out, might be able to compare results with this)
        \item `Design and Implementation of an Ahead-of-Time Compiler for PHP'\cite{Big:10} (to check in detail)
        \footnotetext{http://www.icosaedro.it/phplint, july 2014}
    \end{itemize}
    Also describe their differences with my research.
    
\end{document}