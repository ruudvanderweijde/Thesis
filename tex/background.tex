\documentclass[../main.tex]{subfiles}
\begin{document}
    \chapter{Background and Related Work}\label{ch:background}
    Chapter \ref{ch:background} provides relevant background information.
    The first section, section \ref{sec:background_language-constructs}, describes seven important language constructs which the reader needs to understand in order to understand the difficulties of analysing PHP.
    Section \ref{sec:background_rascal} explains \Gls{Rascal}, the programming language used for the analysis.
    In this section we will explain $M^3$, a programming language independent meta model which holds various facts about programs, in more details.
    Section \ref{sec:background_type-system} provides information about type systems and how type systems relate to this research.
    The last section of this chapter, section \ref{sec:background_related-work}, presents related work and how these researches relate to this thesis.

    \section{PHP Language Constructs}\label{sec:background_language-constructs}
    PHP has various language constructs which complicate \gls{static code analysis}.
    This section presents language constructs and why these constructs are important for this research.
    Explanations of these constructs help you to understand the performed analysis.
    The discussed parts are scope, file includes, conditional classes and functions, dynamic features, late static binding, magic methods and dynamic class properties.
    
    \paragraph{Scope}
    In PHP, all classes and functions are globally accessible once they are declared.
    All classes and functions are implicitly public, inner classes are not allowed, and conditional functions (see upcoming paragraph about conditional classes and functions) will be available in the global scope.
    If a class or function is declared inside a namespace, their full qualified name includes the name of the namespace.
    \\
    Variables have three scope levels: global-, function-, and method-scope.
    Under normal circumstances when a variable is declared inside a function or method, their scope is limited to this function or method.
    Variables declared outside function or methods are available in the global scope, but not in the method or function scope.
    There is an exception for some predefined global variables which are available everywhere. 
    Examples are \texttt{\$GLOBALS}, \texttt{\$\_POST}, and \texttt{\$\_GET}. 
    Variables inside a function or method can be aliased to a global variable by adding the keyword \texttt{GLOBAL} in front of the variable name.
    The variable are then linked to the global variable in the symbol table\footnotemark.
    \footnotetext{http://php.net/manual/en/language.variables.scope.php, July 2016}
    
    \paragraph{Script includes}
    PHP allows files to include other PHP-files during program execution. 
    The content of these files will be loaded at the place where the include statement is defined. 
    This means that if you use an include in the middle of a file, the source code of this included file will be virtually inserted at that position.
    \\
    File includes are mainly used for loading classes and for including templates to render output.
    PHP5 allows automated class loading based on the namespace, which is called autoloading classes.
    With autoloading classes there is no need for including files manually for each class.
    \\
    Research by Mark Hills et al.\cite{Hil:14} has shown that most includes can be resolved with statical analysis.
    In this research we do not run such an analysis, we will assume that all files in the project are included during execution.
    
    \paragraph{Conditional classes and functions}
    Once a file is included in the execution, all the found classes and functions are declared in the top level scope.
    All class and function declarations within condition statements or within a method or function scope are only declared when the conditional statement  is executed.
    \\
    An example of an conditional statement can be found in listing \ref{lst:conditionals1}.
    If the class \texttt{Foo} or function \texttt{bar} do not exist before the statements are executed, then the class and function will not yet be declared. 
    When you try to use the class or function before the code is executed, the script will exit with an fatal error.

    \lstinputlisting[language=PHP,label=lst:conditionals1,caption=Conditional class and function definitions]{src/php/conditionals1.php}

    Another example of dynamic function and class loading is displayed in listing \ref{lst:conditionals2}.
    If the first call is \texttt{g()} as you can see in \texttt{line 8}, the script will result in a fatal error because \texttt{function g()} will only be declared after \texttt{function f()} is executed.
    \texttt{class C} will be declared once \texttt{function g()} is executed.
    As soon as the functions and classes are declared, they are available in the top scope, possibly prefixed with the name of the namespace they are declared within.
    \lstinputlisting[language=PHP,label=lst:conditionals2,caption=Conditional function declaration]{src/php/conditionals2.php}

    
    \paragraph{Dynamic features}
    PHP comes with dynamic built-in features like: include dynamic variables, dynamic class instantiations, dynamic function calls, dynamic function creation, reflection, and eval.
    New functions and classes can be declared on the fly during run-time.
    Method calls, or even whole pieces of code, can be executed based on variable strings.
    \\
    A previous study by Mark Hills\cite{Hil:13} has shown that most real world applications make use of dynamic features.
    Dynamic features are powerful, but can complicate the statical analysis.
    Analysis like constant propagation is needed to help resolving most of these dynamic features.
    This is not in scope for this research, but could be added in future work.
    
    \paragraph{Late static binding}
    Late static binding\footnotemark{} is implemented in PHP since version 5.3 by adding the keyword \texttt{static} to the language.
    \footnotetext{http://php.net/manual/en/language.oop5.late-static-bindings.php, July 2014}
    Its usage is similar to the keyword \texttt{self}, which refers to the current class. 
    The main difference is that \texttt{self} refers to the class where the code is located, while \texttt{static} refers to the actual instantiated class and can be a descendant of the class.
    The keyword \texttt{self} can be statically resolved without running the program while \texttt{static} can only be resolved on runtime.
    
    \paragraph{Magic methods}
    PHP allows calls and property access on methods and fields that don't exist on a class.
    Normally a call to a non-existing method or property would result in a fatal error, but with the use of magic methods you can specify the wanted behaviour.
    Listing \ref{lst:magicMethods} shows an example of the \texttt{\_\_{}call} method.
    This method is triggered when a inaccessible proporties are read.
    In this example the code will try to return the value of a private property based on the provided name.
    The full list of magic methods is \texttt{\_\_{}construct}, \texttt{\_\_{}destruct}, \texttt{\_\_{}call}, \texttt{\_\_{}callStatic}, \texttt{\_\_{}get}, \texttt{\_\_{}set}, \texttt{\_\_{}isset}, \texttt{\_\_{}unset}, \texttt{\_\_{}wakeup}, \texttt{\_\_{}toString}, \texttt{\_\_{}invoke}, \texttt{\_\_{}set\_state}, \texttt{\_\_{}clone}, and \texttt{\_\_{}debugInfo}.
	
	\lstinputlisting[language=PHP,label=lst:magicMethods,caption=Magic methods in PHP]{src/php/magicMethods.php}

    \paragraph{Dynamic class properties}
    Although it is a good practice to define your class properties, it is not required to do so in PHP.
    After instantiating a class it is possible to add properties to objects, even without magic method usage.
    In listing \ref{lst:dynamicProperty} you can see a code sample of adding a property to an object.
    The access of the non-existing property \texttt{nonExistingProperty} will result in a warning, but code execution will continue and will just return \texttt{NULL}.
    The code on \texttt{line 4} creates the property by writing to an nonexisting property.
    The object of variable \texttt{\$c} will have the \texttt{nonExistingProperty} publicly available now.
    But in a new class instantiation, as you can see on \texttt{line 6}, will not have the property.
    
    \lstinputlisting[language=PHP,label=lst:dynamicProperty,caption=Dynamic class property]{src/php/dynamicProperty.php}
    
    \section{Rascal}\label{sec:background_rascal}
    \Gls{Rascal}\cite{Kli:09} is a meta programming language developed by Centrum Wiskunde \& Informatica (CWI).
    Rascal is designed to analyse, transform and visualise source code.
    The language is build on top of Java and implements various concepts of existing programming languages.
    In this research, Rascal is the main programming language.
    Rascal is used for gathering facts about the program and to solve constraints.
    The facts are gathered by visiting AST tree representing the program and hold semantic information about the program.
    Constraints are generated based on the collected facts and these constraints are solved with an in Rascal created constraint solver.
    The only part that does not use Rascal is the PHP parser.
    Although this could be implemented in Rascal, there was an existing library written in PHP available.

    \paragraph{}
    $\pmb{M^3}$\cite{Ana:13,Bas:15} is a model which holds various information of source code and is implemented in Rascal.
    This model is created to gain insights in the quality of open-source projects.
    For our research we use the $M^3$-model to store facts about the program in a structured way, so we can easily use it at a later stage of the analysis.
    
    The core elements of the $M^3$-model are \texttt{containment}, \texttt{declarations}, \texttt{documentation}, \texttt{modifiers}, \texttt{names}, \texttt{types}, \texttt{uses}, \texttt{messages}.
    The \texttt{declarations} relation contains class, method, variable- information with their logical name and their real location. The type of the relation are \texttt{locations} and represent the logical name of the declaration and will be used in the rest of the $M^3$.
    The \texttt{containment} relation has information on what declarations are contained in each other. For example a package can contain a class; a class can contain fields and methods or an inner class; a method can contain variables.    
    The \texttt{documentation} relation contains all comments from the source code and its source location.
    The \texttt{modifiers} relation has information about the modifiers of declarations. Modifiers can be abstract, final, public, protected, or private.
    The \texttt{names} relation contains a simplified name of the full declarations.
    The \texttt{types} relation holds information about the type of the source code elements.
    The \texttt{uses} relation describes what references use an object. For instance when a field of a class is used in some expression, the \texttt{uses} relation links the field in the expression to the declaration of the field in the class.
    And lastly, \texttt{messages} contains global error, warning, and info statements.
    
    
    \section{Type systems}\label{sec:background_type-system}
    A \textbf{type} is a set of possible values and a set of operations that can be performed on them.
    PHP is a dynamically typed language, which means that the types of the expressions are not examined at compile time.
    PHP implements duck typing, so the type of objects are not examined during run-time, but only checks if the operations are allowed on the data.
    In this thesis we are interested in the run-time types of values so we are able to perform static source code analysis.
   
    \paragraph{Type systems} Type systems define how a set of rules are applied to types in their context.
    A type system validates the type usage with \textbf{type checking}.
    In order to perform type checking the types of the expressions needs to be known.
    The process of resolving the types of the expression is called \textbf{type inference}.
    Both type checking and type inference are explained in more details below.
    
    \paragraph{Type checking}
    Type checking is a mechanism which validates and/or enforces the constraints of a type in their specific context.
    There is a difference between static type checking and dynamic type checking.
    \textbf{Static type checking} is a process of checking the types based on the source code.
    The static type checker will ensure that a program is type safe before executing the program, which means that there will occur no type errors during runtime.
    \textbf{Dynamic type checking} performs the type checking during runtime.
    This means that the program needs to run to gain feedback on the usage of types.
    PHP is a dynamically typed language, which means that there are no types checked before actually running the program.
    Although PHP also implements some static type features, like parameter type hints, it cannot be perfectly determine all types at compile time.
        
    \paragraph{Type inference}
    Type inference is the process of resolving types of variables and expressions.
    The inference process is a prerequisite to perform type checking.
    Being able to infer the type before running the program enables you to optimise code execution by applying compiler optimisations.
    These optimisation can be performance improvements or memory optimisations.
    In dynamic languages like PHP, it can be hard to resolve the type of a variable or expression without running the program.
    In statically typed languages, type inference happens at compile time.
    In the next paragraph we will briefly explain some type inference systems.
    
    The \textbf{Hindley-Milner}\cite{Hin:69} (HM) type system was found in 1969 by Roger J. Hindley and almost 10 years later rediscovered\cite{Mil:78} by Robin Milner.
    The first implementation was created four years later by pHd student Luis Damas.
    Damas proved the soundness and completeness of the HM type system with \textbf{Algorithm W}\cite{Dam:82}  in the context of the programming language \texttt{ML}.
    The HM type system deduces the types of the variables to their most abstract type, based on their usage.
    Type declarations and hints are not necessarily to perform type inference.
    The type system is used for various functional languages. 
    \texttt{Haskell} for example uses the Hinley-Milner type system as a foundation for the \texttt{Haskell} type system. 
    
    \textbf{Control Flow Analysis}\cite{Nie:99} (CFA) is concerned with resolving sound approximate run-time values at compile time.
    CFA is build on top of data flow analysis\cite{Aho:86} and tries to resolve the control-flow problem for high order programming languages.
    The control-flow problem deals with resolving which caller can call which callee in a program.
    One of the earlier CFA algorithms was Shivers' $0$CFA algorithm\cite{Shi:88}, a flow-sensitive constraint based algorithm.
    Shiver then defined $k$-CFA\cite{Shi:91}, where the precision of the analysis is increased by taking the context of the expressions into account.
    The $k$-CFA algorithms compute an conservative over-approximations of run-time values during compile type.
	
    The \textbf{Cartesian Product Algorithm}\cite{Age:95} (CPA) is a type inference algorithm created by Ole Agesen in 1995.
    Agesens work was based on Palsberg and Schwartzbach' \textbf{basic type inference algorithm}\cite{Pal:91}.
    This basic type inference algorithm derives a set of constraints based on \textit{trace graphs} and solves the constraints using a fix-point algorithm.
    Agesen extended the basic algorithm with \textit{templates}.
    These templates are based on control flow and have start and end nodes with their possible in- and outputs.
    The CPA calculates the possibles output types for each template by taking the cartesian product, the set of all possible ordered pairs, of the input types.

    \section{Related work}\label{sec:background_related-work}
    In this section we briefly describe related research work in order to get a better understanding of similar performed researches.
    
    \paragraph{}
    Similar work has been presented by Patrick Camphuijsen\cite{Cam:07, Cam:09}.
    Patrick created a constraint-based type inference analysis for his master's thesis.
    The inference algorithm combines possible results of the constraints and takes the union to define the types.
    To guarantee termination the algorithm uses widening, by replacing the current result with the result of the union, to make sure that there will be a fixed-point.
    Further work improved the implementation by adding support for objects\cite{Hoe:15}.
    
    \paragraph{}
    Paul Biggar created an Ahead-Of-Time (AOT) compiler for PHP\cite{Big:10}.
    The main goal of this compiler is to improve the performance of PHP programs.
    The AOT compiler starts by parsing a PHP program into an AST.
    This AST is transformed into an High-level Intermediate Representation (HIR) to remove all redundant constructs and then transformed into a Medium-level Intermediate Representation (MIR).
    Using dataflow analysis, alias analysis, static single assignment (SSA), and type analysis the compiler performs optimisations on the MIR.
    After the optimisations, the compiler generates C code, which then can be executed to run the program.
   
    \paragraph{}
    \texttt{PHANTM}\cite{Kne:10,Bar:10} (PHp ANalyzer for Type Mismatch) is an open source PHP analyser written in Scala.
    Because of PHP's dynamic nature, without compiler or interpreter type checking, it is easy to make typing errors that result in unexpected behaviour or in fatal errors.
    \texttt{PHANTM} performs a hybrid flow-sensitive analysis to find type errors in PHP5.
    The hybrid analysis combines static and dynamic analysis.
    A program can be annotated to start a static analysis at a specific point.
    The analyser collects run-time type information while running the program and then starts the static analysis.
    \texttt{PHANTM} uses data-flow analysis to infer types.
    Although \texttt{PHANTM} has proven to be able to find a decent number of type errors on scalar usages in three different programs, there is a lack of finding errors in object oriented structures.
    
    \paragraph{}
    Facebook improved the performance of PHP programs with a static compiler, called HipHop Virtual Machine\cite{Zha:12} (HHVM).
    This static compiler extracts the program into an AST, traverses this AST to collect information, performs pre-optimisations, performs type inference, preforms post-optimisations, and lastly generates C++ code.
    During the pre-optimisations the compiler removes unneeded actions, for example constant inlining, logical-expression simplifications, and dead-code elimination.
    The type inference process is based on the Hindley-Milner constraint based algorithm\cite{Dam:82}, to infer types of constants, variables, functions parameters, and return types.
    These new inferred types are then used in the post-optimisation.
    In the last step the AST is traversed to generate C++ code.
    Although the compiler does not cover all functions of PHP, it does covers most of the features.
    The performance benefits on the other hand are significantly better, showing on average 5.5x more efficiency for PHP5.
        
    \paragraph{}
    PHPLint\footnotemark{} extends the PHP syntax with type hints where PHP lacks support for it, using custom inline comment blocks to add extra typing information.
    \footnotetext{http://www.icosaedro.it/phplint, July 2016}
    These doc blocks with type information can be used in the analysis, allowing more strict type checking.
    The used syntax for the type hints are \texttt{/*. .*/}, for example: \texttt{/*. string .*/ \$s = \$var;} which means that the variable \texttt{\$s} is of type \texttt{string}.
    PHPLint solves the lack of type hint support on scalar and array types in PHP5.
    PHPLint can generate type hints based on information retrieved from simple type inference.

\end{document}