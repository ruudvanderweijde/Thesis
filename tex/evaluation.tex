\documentclass[../main.tex]{subfiles}
\begin{document}
    \chapter{Evaluation}\label{ch:evaluation}
    % todo update introduction of this chapter to include a line about the content
	In this chapter we evaluate the steps we took in this research.
	We first describe the setup of the experiment in section \ref{sec:evaluation_experiment_setup}.
	Next we present the results in section \ref{sec:evaluation_results}.
	In section \ref{sec:evaluation_analysis} we analyse the measured results.
	
	% Methode: Hoe zet je het experiment op en wat ga je meten
	\section{Experiment setup}\label{sec:evaluation_experiment_setup}
	
	In order to validate the this research we tested our type inference algorithm on a selection of the 30 most popular packages of Packagist\footnotemark.
	\footnotetext{https://packagist.org/explore/popular, July 2014}
	Packagist is a repository for Composer\footnotemark projects.
	\footnotetext{https://getcomposer.org/, July 2014}
	Composer is a dependency manager for PHP.
	All projects have between 2 and 6 million downloads, so they should be suitable for our research because they are frequently used in live applications.
	The selection of projects, sorted on total lines of code, is listed in table \ref{table:corpus}.
	% Due to performance reasons we are only able to analyse the smaller projects.
	The statistics are generated using phploc\footnotemark.
	\footnotetext{https://github.com/sebastianbergmann/phploc, July 2014}

	
\npaddmissingzero
\npfourdigitsep
\begin{table}[H]
  \centering
  \scriptsize
  \begin{tabular}{@{}lllrrrrrrrr@{}} \toprule
     \multicolumn{3}{c}{Product}        & \multicolumn{2}{c}{Files} & \multicolumn{3}{c}{Objects}        & \multicolumn{3}{c}{Lines of code} \\
     \cmidrule(l{2pt}r{2pt}){1-3}       \cmidrule(l{2pt}r{2pt}){4-5} \cmidrule(l{2pt}r{2pt}){6-8}        \cmidrule(l{2pt}r{2pt}){9-11}                    
     Vendor & Project* & Version           & D$^1$         & F$^2$          & C$^3$ & I$^4$ & T$^5$ & Total $\uparrow$ & \multicolumn{2}{c}{Logical} \\ \midrule
doctrine & lexer & v1.0 & 2 & 7 & \numprint{3} & \numprint{0} & \numprint{0} & \numprint{733} & \numprint{128} & (17.46\%) \\
phpunit & php-timer & 1.0.5 & 5 & 11 & \numprint{5} & \numprint{0} & \numprint{0} & \numprint{740} & \numprint{117} & (15.81\%) \\
phpunit & php-text-template & 1.2.2 & 5 & 11 & \numprint{5} & \numprint{0} & \numprint{0} & \numprint{768} & \numprint{125} & (16.28\%) \\
doctrine & inflector & v1.0 & 2 & 7 & \numprint{3} & \numprint{0} & \numprint{0} & \numprint{853} & \numprint{130} & (15.24\%) \\
psr-fig & log & 1.0.0 & 3 & 15 & \numprint{8} & \numprint{2} & \numprint{2} & \numprint{1039} & \numprint{155} & (14.92\%) \\
phpunit & php-file-iterator & 1.3.4 & 5 & 13 & \numprint{7} & \numprint{0} & \numprint{0} & \numprint{1071} & \numprint{176} & (16.43\%) \\
%symfony & filesystem & v2.5.3 & 3 & 11 & \numprint{5} & \numprint{2} & \numprint{0} & \numprint{1090} & \numprint{193} & (17.71\%) & \numprint{19} & (9.84\%) \\
%symfony & yaml & v2.5.3 & 3 & 16 & \numprint{11} & \numprint{1} & \numprint{0} & \numprint{2270} & \numprint{509} & (22.42\%) & \numprint{28} & (5.50\%) \\
%phpunit & php-token-stream & 1.2.2 & 6 & 13 & \numprint{169} & \numprint{0} & \numprint{0} & \numprint{2360} & \numprint{377} & (15.97\%) & \numprint{15} & (3.98\%) \\
%doctrine & collections & v1.2 & 3 & 18 & \numprint{11} & \numprint{3} & \numprint{0} & \numprint{2504} & \numprint{394} & (15.73\%) & \numprint{33} & (8.38\%) \\
%symfony & process & v2.5.3 & 3 & 19 & \numprint{14} & \numprint{1} & \numprint{0} & \numprint{3198} & \numprint{604} & (18.89\%) & \numprint{37} & (6.13\%) \\
%symfony & finder & v2.5.3 & 8 & 43 & \numprint{36} & \numprint{3} & \numprint{0} & \numprint{4976} & \numprint{909} & (18.27\%) & \numprint{80} & (8.80\%) \\
%symfony & dom-crawler & v2.5.3 & 12 & 63 & \numprint{53} & \numprint{6} & \numprint{0} & \numprint{7825} & \numprint{1296} & (16.56\%) & \numprint{157} & (12.11\%) \\
%symfony & translation & v2.5.3 & 21 & 121 & \numprint{97} & \numprint{20} & \numprint{0} & \numprint{12345} & \numprint{2299} & (18.62\%) & \numprint{257} & (11.18\%) \\
%symfony & console & v2.5.3 & 17 & 84 & \numprint{66} & \numprint{13} & \numprint{2} & \numprint{13546} & \numprint{2556} & (18.87\%) & \numprint{246} & (9.62\%) \\
%symfony & http-foundation & v2.5.3 & 16 & 90 & \numprint{76} & \numprint{10} & \numprint{0} & \numprint{14179} & \numprint{2262} & (15.95\%) & \numprint{154} & (6.81\%) \\
%twig & twig & v1.16.0 & 18 & 172 & \numprint{148} & \numprint{19} & \numprint{0} & \numprint{14689} & \numprint{2630} & (17.90\%) & \numprint{15} & (0.57\%) \\
%symfony & event-dispatcher & v2.5.3 & 27 & 170 & \numprint{133} & \numprint{31} & \numprint{3} & \numprint{20230} & \numprint{3629} & (17.94\%) & \numprint{418} & (11.52\%) \\
%swiftmailer & swiftmailer & v5.2.1 & 37 & 238 & \numprint{170} & \numprint{52} & \numprint{0} & \numprint{28965} & \numprint{4645} & (16.04\%) & \numprint{144} & (3.10\%) \\
%phpunit & php-code-coverage & 2.0.1 & 62 & 259 & \numprint{381} & \numprint{24} & \numprint{0} & \numprint{50371} & \numprint{6579} & (13.06\%) & \numprint{87} & (1.32\%) \\
%phpunit & phpunit & 4.2.2 & 65 & 270 & \numprint{388} & \numprint{26} & \numprint{0} & \numprint{51516} & \numprint{6764} & (13.13\%) & \numprint{129} & (1.91\%) \\
%phpunit & phpunit-mock-objects & 2.2.0 & 66 & 271 & \numprint{393} & \numprint{27} & \numprint{0} & \numprint{51735} & \numprint{6801} & (13.15\%) & \numprint{132} & (1.94\%) \\
%doctrine & annotations & v1.2.0 & 69 & 306 & \numprint{423} & \numprint{28} & \numprint{0} & \numprint{57325} & \numprint{7718} & (13.46\%) & \numprint{188} & (2.44\%) \\
%doctrine & common & v2.4.2 & 76 & 337 & \numprint{440} & \numprint{45} & \numprint{0} & \numprint{62406} & \numprint{8326} & (13.34\%) & \numprint{298} & (3.58\%) \\
%symfony & http-kernel & v2.5.3 & 96 & 565 & \numprint{471} & \numprint{90} & \numprint{3} & \numprint{79294} & \numprint{14169} & (17.87\%) & \numprint{1449} & (10.23\%) \\
%doctrine & cache & 1.3.0 & 152 & 687 & \numprint{729} & \numprint{102} & \numprint{2} & \numprint{103024} & \numprint{16667} & (16.18\%) & \numprint{1355} & (8.13\%) \\
%doctrine & dbal & v2.4.2 & 121 & 557 & \numprint{628} & \numprint{63} & \numprint{0} & \numprint{104630} & \numprint{15234} & (14.56\%) & \numprint{1033} & (6.78\%) \\
%guzzle & guzzle & v3.9.2 & 150 & 832 & \numprint{828} & \numprint{141} & \numprint{7} & \numprint{117699} & \numprint{19772} & (16.80\%) & \numprint{1787} & (9.04\%) \\
%doctrine & orm & v2.4.4 & 175 & 1007 & \numprint{875} & \numprint{119} & \numprint{2} & \numprint{158530} & \numprint{27932} & (17.62\%) & \numprint{2866} & (10.26\%) \\
%monolog & monolog & 1.10.0 & 350 & 1911 & \numprint{1904} & \numprint{135} & \numprint{2} & \numprint{288507} & \numprint{31415} & (10.89\%) & \numprint{4221} & (13.44\%) \\
%werkspot & old-Website & 07-2014 & 928 & 6225 & \numprint{4907} & \numprint{224} & \numprint{0} & \numprint{1054686} & \numprint{167978} & (15.93\%) & \numprint{22693} & (13.51\%) \\
  \bottomrule
     \multicolumn{11}{l}{} \\
     \multicolumn{11}{l}{*Selection of the 30 most popular packagist packages ordered by LOC, in July 2014.} \\
     \multicolumn{11}{l}{$^1$ = Directories, $^2$ = Files, $^3$ = Classes, $^4$ = Interfaces, $^5$ = Traits} \\
  \end{tabular}
  \normalsize
  \caption{Statictics of most populair composer projects\label{table:corpus}}
\end{table}
\npfourdigitnosep
\npnoaddmissingzero

	\paragraph{Type inference}
	In this research we are interested in resolving types for expressions.
	As explained in section \ref{sec:implementation:constraint_solving}, the result of the constraint solving is a set of types for each expression.
	We group these resulting typesets in two groups, resolved and unresolved types.
	Resolved type means that we can deduce the typeset to one type.
	All other types are unresolved.
	In unresolved we include typesets that are not resolved and have only type \texttt{any()}, empty typesets, and multiple types that cannot be reduced to one unique type for an expression.
	
	\paragraph{Annotations}
	We first run the type inference without reading annotations from PHPDocs.
	As described in section \ref{sec:design_annotations}, we can use type annotations to read types of variables, functions, methods, and class attributes.
	Our goal is to see if it is possible to resolve more types when we take the type annotations from the PHPDocs into account.
	
	\paragraph{Built-ins}
	After we run the analysis with and without PHPDocs, we run the analysis also with built-in information.
	Modern editors stub files written in PHP which represent the internal behaviour of constants, variables, functions, classes, methods and class attributes.
	Using the type information of the built-in language constructs we want to see if we are able to resolve more types.
			
	% Resultaten: met tabellen en zo
	\section{Results}\label{sec:evaluation_results}
	
	In this section we present the results of the analysis.
	We start with the results of the analysis with and without taking PHPDocs into account.
	Next we show the results of the analysis with and without PHP built-ins typehints.

	\paragraph{Annotations}
	The results of the analysis without and without taking type annotations from PHPDocs into account are shown in table \ref{table:results:rascal_results_docblocks}.
	In this table on the left side we present the name of the project and the total number of expressions that were attempted to be inverted.
	On the right side of the table we present the percentage of resolved types for the analysis with and without taking the PHPDoc context into account.

%% The tables below is generated using: `lang::php::experiments::mscse2014::RascalResultAnalysis`
\npaddmissingzero
\npfourdigitsep
\begin{table}[H]
        \centering
        \scriptsize
        \begin{tabular}{@{}lr|rr@{}} 
                \toprule
                        & &
                        \multicolumn{2}{c}{Resolved types} \\

                        Project & Total &
                        w/o PHPDoc &
                        w/ PHPDocs \\
                \midrule
                        php-timer &
                        \numprint{68} & % total
                        \numprint{21.4}\% & \numprint{64.3}\% \\ 
                        log &
                        \numprint{120} & % total
                        \numprint{40.9}\% & \numprint{40.9}\% \\ 
                        php-text-template &
                        \numprint{77} & % total
                        \numprint{34.8}\% & \numprint{43.5}\% \\ 
                        lexer &
                        \numprint{96} & % total
                        \numprint{45.2}\% & \numprint{57.1}\% \\ 
                        inflector &
                        \numprint{85} & % total
                        \numprint{9.7}\% & \numprint{45.2}\% \\ 
                        php-file-iterator &
                        \numprint{117} & % total
                        \numprint{30.2}\% & \numprint{44.4}\% \\ 
                \bottomrule
        \end{tabular}
        \normalsize
\caption{Type inference results, with and without PHPDocs\label{table:results:rascal_results_docblocks}}
\end{table}
\npfourdigitnosep
\npnoaddmissingzero
	
	\paragraph{Built-ins}
	The results of type inference where we take type information of PHP built-in functions and variables into account are shown in table \ref{table:results:rascal_results_built-ins}.
	In this table we present the analysed projects and the total amount of expressions we attempted to infer.
	On the right side of the table we present the percentage of resolved types for the analysis with and without built-in information.
	
\npaddmissingzero
\npfourdigitsep
\begin{table}[H]
        \centering
        \scriptsize
        \begin{tabular}{@{}lr|rr@{}} 
                \toprule
                        & &
                        \multicolumn{2}{c}{Resolved types} \\

                        Project & Total &
                        w/o Built-ins &
                        w/ Built-ins \\
                \midrule
                        php-timer &
                        \numprint{68} & % total
                        \numprint{64.3}\% & \numprint{88.2}\% \\ 
                        log &
                        \numprint{120} & % total
                        \numprint{40.9}\% & \numprint{65.0}\% \\ 
                        php-text-template &
                        \numprint{77} & % total
                        \numprint{43.5}\% & \numprint{79.2}\% \\ 
                        lexer &
                        \numprint{96} & % total
                        \numprint{57.1}\% & \numprint{78.1}\% \\ 
                        inflector &
                        \numprint{85} & % total
                        \numprint{45.2}\% & \numprint{76.5}\% \\ 
                        php-file-iterator &
                        \numprint{117} & % total
                        \numprint{44.4}\% & \numprint{68.4}\% \\ 
                \bottomrule
        \end{tabular}
        \normalsize
\caption{Results of type usage, with and without PHP built-ins\label{table:results:rascal_results_built-ins}}
\end{table}
\npfourdigitnosep
\npnoaddmissingzero


	
	% Analyse: hoe verklaar je wat we zien in de tabellen en klopt het met de theorie?
	\section{Analysis}\label{sec:evaluation_analysis}
	In the previous section we presented the number of resolved expressions.
	To get a better insight in the seen an overal increase of resolved types when adding more information to the type inference.
	When adding type annotations from PHPDocs we've seen a big increase of resolved types.
	
	% some graph here where we show the difference
	
	% some text I removed from the previous chapter
	%We see in table \ref{table:results:rascal_results_php_internals} an overal increase of resolved types when using the information of PHP built-in.
	%todo explain more here.
	%We do however also see a small decrease in the number of inferred types for two projects.
	

\pgfplotstableread[col sep=comma,header=false]{
        php-timer,3,9,60
        log,27,27,78
        php-text-template,8,10,61
        lexer,19,24,75
        inflector,3,14,65
        php-file-iterator,19,28,80
}\data
\pgfplotsset{
        percentage plot/.style={
                point meta=explicit,
                yticklabel=\pgfmathprintnumber{\tick}\,$\%$,
                ymin=0,
                ymax=100,
                enlarge y limits={upper,value=0},
                visualization depends on={y \as \originalvalue}
        },
        percentage series/.style={
                table/y expr=\thisrow{#1},table/meta=#1
        }
}
\begin{tikzpicture}
\begin{axis}[
        axis on top,
        width=16cm,
        height=7cm,
        ylabel=Percentage of resolved items,
        xlabel=,
        percentage plot,
        ybar,
        bar width=0.5cm,
        enlarge x limits=0.12,
        cycle list={
                {fill=black!10,draw=black,postaction={pattern=crosshatch dots}},
                {fill=black!30,draw=black,postaction={pattern=north east lines}},
                {fill=black!50,draw=black,postaction={pattern=crosshatch}}
        },
        legend entries={entry1,entry2},
        legend image code/.code={%
                \draw[#1] (0cm,-0.1cm) rectangle (0.6cm,0.1cm);
        },
        major grid style=white,
        symbolic x coords={php-timer,log,php-text-template,lexer,inflector,php-file-iterator},
        xtick=data,
        nodes near coords,
        nodes near coords align={vertical},
        x tick label style={rotate=45,anchor=east}
]
\addplot table [percentage series=1] {\data};
\addplot table [percentage series=2] {\data};
\addplot table [percentage series=3] {\data};
\legend{normal,w/ phpdoc, w/ built-ins}
\end{axis}
\end{tikzpicture}
	
	
	
	
\pgfplotstableread[col sep=comma,header=false]{
        php-timer,21,64,88
        log,40,40,65
        php-text-template,34,43,79
        lexer,45,57,78
        inflector,9,45,76
        php-file-iterator,30,44,68
}\data
\pgfplotsset{
        percentage plot/.style={
                point meta=explicit,
                yticklabel=\pgfmathprintnumber{\tick}\,$\%$,
                ymin=0,
                ymax=100,
                enlarge y limits={upper,value=0},
                visualization depends on={y \as \originalvalue}
        },
        percentage series/.style={
                table/y expr=\thisrow{#1},table/meta=#1
        }
}
\begin{tikzpicture}
\begin{axis}[
        axis on top,
        width=16cm,
        height=7cm,
        ylabel=Percentage of resolved items,
        xlabel=,
        percentage plot,
        ybar,
        bar width=0.5cm,
        enlarge x limits=0.12,
        cycle list={
                {fill=black!10,draw=black,postaction={pattern=crosshatch dots}},
                {fill=black!30,draw=black,postaction={pattern=north east lines}},
                {fill=black!50,draw=black,postaction={pattern=crosshatch}}
        },
        legend entries={entry1,entry2},
        legend image code/.code={%
                \draw[#1] (0cm,-0.1cm) rectangle (0.6cm,0.1cm);
        },
        major grid style=white,
        symbolic x coords={php-timer,log,php-text-template,lexer,inflector,php-file-iterator},
        xtick=data,
        nodes near coords,
        nodes near coords align={vertical},
        x tick label style={rotate=45,anchor=east}
]
\addplot table [percentage series=1] {\data};
\addplot table [percentage series=2] {\data};
\addplot table [percentage series=3] {\data};
\legend{normal,w/ phpdoc, w/ built-ins}
\end{axis}
\end{tikzpicture}


\end{document}