\documentclass{uvamscse}

\usepackage{sty/bussproofs}
\usepackage{sty/custom}
\usepackage{sty/rascal}
\usepackage{sty/codeformat}

%\title{Value of annotations in PHP}
\title{Type inference for PHP}
%\subtitle{The value of annotations in a dynamic language}
\subtitle{A constraint based type inference written in Rascal}

\author{Ruud van der Weijde}

\supervisor{Jurgen Vinju}
\host{Werkspot, \url{http://www.werkspot.nl}}
\supervisorhost{Winfred Peereboom}

\abstract{
Dynamic language like PHP are generally hard to statically analyse because of run-time dependencies.
Despite the wide usage of PHP programs over the internet, there seem not many tools available to support all aspect completely.
In statical analysis the program is analysed without running the program and there are many things undecided.
In this thesis we present a constraint based type inference written in Rascal.
Rascal is a programming language for meta-programming in the domain of software analysis and transformations.
We created this type inference for PHP to be able to resolve the types of expressions used in programs.
Follow up analysis can be performed when expressions are typable, for example to find vulnerabilities or provide programming context in IDEs.
In a small experiment where we tested if adding type annotations of PHP docblock and PHP build-in information would help to infer more types.  
We saw that the number of inferred types increases when type hint annotations are taken into account.
}


\begin{document}
    \maketitle
    
	% contains glossary declarations, can be used using \gls{key} or \Gls{key}
    \makeglossaries
	\subfile{tex/glossaryDef.tex}
    
   	\subfile{tex/preface.tex}
   	\subfile{tex/introduction.tex}
	\subfile{tex/background.tex}
   	\subfile{tex/research_method.tex}
   	\subfile{tex/inference_design.tex}
   	\subfile{tex/inference_implementation.tex}
   	\subfile{tex/evaluation.tex}
    \subfile{tex/analysis.tex}
    %\subfile{tex/results.tex}
    %\subfile{tex/case_study.tex}
   	\subfile{tex/conclusion.tex}
   	%\subfile{tex/test.tex}
    \printglossary[style=altlist]
	\printbibliography
    
\end{document}
