\documentclass{uvamscse}

\usepackage{sty/bussproofs}
\usepackage{sty/custom}
\usepackage{sty/rascal}
\usepackage{sty/codeformat}

%\title{Value of annotations in PHP}
\title{Type inference for PHP}
%\subtitle{The value of annotations in a dynamic language}
\subtitle{A constraint based type inference written in Rascal}

\author{Ruud van der Weijde}

\supervisor{Jurgen Vinju}
\host{Werkspot, \url{http://www.werkspot.nl}}
\supervisorhost{Winfred Peereboom}


\abstract{
Despite the wide usage of PHP programs over the internet, there is a lack of good analysis tools.
Dynamic language like PHP are generally hard to statically analyse because of run-time dependencies. 
Without running the program there are many things undecided.
In this thesis we present a constraint based type inference written in Rascal.
Rascal is a programming language for meta-programming in the domain of software analysis and transformations.
We show that the number of inferred types increases when type hint annotations are taken into account.
}


\begin{document}
    \maketitle
    
	% contains glossary declarations, can be used using \gls{key} or \Gls{key}
    \makeglossaries
	\subfile{tex/glossaryDef.tex}
    
   	\subfile{tex/preface.tex}
   	\subfile{tex/introduction.tex}
	\subfile{tex/background.tex}
   	\subfile{tex/research_method.tex}
   	\subfile{tex/inference_design.tex}
   	\subfile{tex/inference_implementation.tex}
   	\subfile{tex/evaluation.tex}
    \subfile{tex/analysis.tex}
    %\subfile{tex/results.tex}
    %\subfile{tex/case_study.tex}
   	\subfile{tex/conclusion.tex}
   	%\subfile{tex/test.tex}
    \printglossary[style=altlist]
	\printbibliography
    
\end{document}
