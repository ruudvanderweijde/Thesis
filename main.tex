\documentclass{uvamscse}

\usepackage{sty/bussproofs}
\usepackage{sty/custom}
\usepackage{sty/rascal}
\usepackage{sty/codeformat}
\usepackage[firstpage]{draftwatermark}
\SetWatermarkScale{3}
\SetWatermarkText{First draft}

%\title{Value of annotations in PHP}
\title{Type inference for PHP}
\subtitle{The value of annotations in a dynamic language}

\author{Ruud van der Weijde}

\supervisor{Jurgen Vinju}
\host{Werkspot, \url{http://www.werkspot.nl}}
\supervisorhost{Winfred Peereboom}


\abstract{
Dynamic language are generally hard to statically analyse because of run-time dependencies. 
Without running the program there are many things unknown.
Because dynamic languages are PHP are widely used, the need for decent analysis tool grows.
This research examines the value of adding annotations to PHP code to improve the analysability.
In the results we see that annotations improve the analysability of software code (this is a guess).
Here I should state something about the correctness of the annotations.
And end with a general conclusion.
}


\begin{document}
    \maketitle
	\makeglossaries
    \subfile{tex/glossaryDef.tex} % contains glossary declarations, can be used using \gls{key} or \Gls{key}
    
   	\subfile{tex/preface.tex}
   	\subfile{tex/introduction.tex}
	\subfile{tex/background.tex}
   	\subfile{tex/research_method.tex}
   	\subfile{tex/research.tex}
    \subfile{tex/analysis.tex}
    \subfile{tex/results.tex}
    \subfile{tex/case_study.tex}
   	\subfile{tex/conclusion.tex}
   	%\subfile{tex/test.tex}
    \printglossary[style=altlist]
	\printbibliography
    
\end{document}
