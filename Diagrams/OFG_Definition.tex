\documentclass[preview,border=20pt,varwidth]{standalone} 
\usepackage{syntax}
\usepackage{changepage}

\setlength{\grammarparsep}{0.1cm}   % vertical distance between production rules
\setlength{\grammarindent}{1.1cm}      % horizontal indent distance

\renewcommand{\syntleft}{}          % do not display '<' associated with variable, for example <A>
\renewcommand{\syntright}{}         % do not display '>' associated with variable, for example <A>


\usepackage{etoolbox} % for patching
\makeatletter
% define the main command on the model of the original one
% we add stepping the counter and typesetting the number
\def\gr@implnumbereditem<#1> #2 {%
  \stepcounter{grammarline}%
  \sbox\z@{\hskip\labelsep\grammarlabel{#1}{#2}}
  \strut\@@par%
  \vskip-\parskip%
  \vskip-\baselineskip%
  \hrule\@height\z@\@depth\z@\relax%
  \item[%
    \rlap{\hskip\dimexpr\linewidth+\grammarindent\relax %% add the number
          \llap{(\thegrammarline)}}%
    \unhbox\z@]%
  \catcode`\<\active%
}
% copy the grammar environment under a new name
\let\numberedgrammar\grammar
\let\endnumberedgrammar\endgrammar
% now patch the new environment
\pretocmd\numberedgrammar{\setcounter{grammarline}{0}}{}{}
\patchcmd\numberedgrammar
  {\gr@implitem}
  {\gr@implnumbereditem}
  {}{}
\patchcmd\numberedgrammar
  {\def\alt{\\\llap{\textbar\quad}}}
  {\let\alt\alt@num}
  {}{}

% the command for numbering the \alt lines
\def\alt@num{\\\relax
  \stepcounter{grammarline}%
  \rlap{\hskip\dimexpr\linewidth-\labelwidth+\grammarindent-\labelsep\relax
        \llap{(\thegrammarline)}}% add the number
  \llap{\textbar\quad}}

\newcounter{grammarline}
\makeatother



\begin{document}

\begin{numberedgrammar}

<P> ::= <D>* <S>*

<D> ::= a
\alt $m(f_1,...,f_k)$
\alt $cs(f_1,...,f_k)$
\alt $f(f_1,...,f_k)$

<S> ::= $x = new$ $c($[$a_1, ..., a_k$]$);$
\alt $x = y;$
\alt [$x =$] $y.m($[$a_1, ..., a_k$]$);$
\alt [$x =$] $f($[$a_1, ..., a_k$]$);$
\end{numberedgrammar}

\noindent
\underline{Legend:} \\
\noindent
Metasymbols: * (repetition), $|$ (alternative), [ ] (optional part) \\
Non terminals: upper case letters \\
Fully scoped identifiers: lower case letters \\
Terminals: all the other symbols \\
\\
\underline{Class scope identifiers:} \\
\begin{small}
\begin{tabular}{ r l l }
    $a$: & class attribute name & $<attr>$ \\
    $m$: & method name & $<meth>$ \\
    $f_1, ..., f_k$: & formal parameters & $<param>$ \\
    $x,y$: & program locations & $<progloc>$ \\
    $a_1, ..., a_1$: & actual parameters & $<progloc>$ \\
    $cs$: & class constructor & $<constr>$ \\
    $c$: & class name & $<class>$ \\
    $f$: & function name & $<function>$ \\
\end{tabular}
\end{small}
\\ \\
where: \\
\begin{adjustwidth}{-0.4cm}{}
\begin{small}
\begin{tabular}{ r l l }
    $<attr>$ & attribute & [$<nspref>$] $<cpref>$ $<vid>$ \\
    $<meth>$ & method & [$<nspref>$] $<cpref>$ $<mid>$ \\
    $<param>$ & parameter & [$<nspref>$] $<cpref>$ $<mid>$ $.$ $<vid>$ \\
    $<constr>$ & class constructor & [$<nspref>$] $<cpref>$ $<cid>$ \\
    $<class>$ & class name & [$<nspref>$] $<cid>$ \\
    $<locvar>$ & variable & [$<nspref>$] [$<cpref>$ $<mid>$ $|$ $<fid>$] $<vid>$ \\
    $<progloc>$ & program location & [$<nspref>$] $<cpref>$ $<vid>$ \\
    $<nspref>$ & namespace prefix & $<pid>$ \\
    $<pid>$ & package identifier & \\
    $<cid>$ & class identifier & \\
    $<mid>$ & method identifier & \\
    $<fid>$ & function identifier & \\
    $<vid>$ & variable identifier & \\
\end{tabular}
\end{small}
\end{adjustwidth}


\end{document}